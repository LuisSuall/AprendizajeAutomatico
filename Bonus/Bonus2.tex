%Encabezado estándar
\documentclass[10pt,a4paper]{article}
\usepackage[utf8]{inputenc}

\usepackage{amsmath}
\usepackage{amssymb}
\usepackage{amsthm}
\usepackage{amsfonts}
\usepackage{enumerate}
\usepackage{listings}
\usepackage{hyperref}

\author{Luis Suárez Lloréns}
\title{Bonus 2}
\date{}

\theoremstyle{definition}
\newtheorem*{cuestion}{Cuestión}
\newtheorem*{respuesta}{Respuesta}

\begin{document}
\maketitle

\section*{Regresión logística}
\begin{cuestion}
\textbf{4.} Consideremos el caso de la verificación de la huella digital (ver transparencias de clase). Tras aprender con un modelo de regresión logística a partir de datos obtenemos una función una hipótesis final
    \[
    g(x)=\mathbb{P}[y=+1|\mathbf{x}]
    \]
    que representa la estimación de la probabilidad de que $y=+1$. Suponga que la matriz de coste está dada por 
    \vspace{10pt}
    
    \centerline{
    \begin{tabular}{c r| c c}
         &  &  \multicolumn{2}{c} {Verdadera Clasificación}\\
         &  &  +1 (persona correcta) & -1 (intruso)\\\hline
         decisión  & {+1} & 0 & $c_a$ \\
         decisión & {-1} & $c_r$ & 0 \\\\
    \end{tabular}}
Para una nueva persona con huella digital $\mathbf{x}$, calculamos $g(\mathbf{x})$ y tenemos que decidir si aceptar o rechazar a la persona ( i.e. tenemos que usar una decisión 1/0). Por tanto aceptaremos si $g(\mathbf{x})\geq \kappa$, donde $\kappa$ es un umbral.
\begin{enumerate}
    \item Definir la función de costo(aceptar) como el costo esperado si se acepta la persona. Definir de  forma similar el costo(rechazo). Mostrar que
     \vspace{10pt}
     
    \centerline{
    \begin{tabular}{rcc}
    costo(acceptar) & = & $(1-g(\mathbf{x}))c_a$\\
     costo(rechazar) & = & $g(\mathbf{x})c_r$ \\
    \end{tabular}}
    \item Usar el apartado anterior para derivar una condición sobre $g(x)$ para aceptar la persona y mostrar que
    \[
    \kappa=\frac{c_a}{c_a+c_r}
    \]
    \item Usar las matrices de costo para la aplicación del supermercado y la CIA (transparencias de clase) para calcular el umbral $\kappa$ para cada una de las dos clases. Dar alguna interpretación del umbral obtenido.
\end{enumerate}
\end{cuestion}
\newpage
\begin{respuesta}
\begin{enumerate}
\hfill \break
\item Vamos a ver las funciones costo suponiendo que hemos aceptado o que hemos rechazado a la persona. Si aceptamos correctamente, la función costo vale 0, y la probabilidad de que esto pase es $g(x)$. Si aceptamos incorrectamente, la función de coste vale $c_a$ y la probabilidad es la contraria a aceptar correctamente, $1-g(x)$. Por tanto, el costo esperado de aceptar es:
\[
\ \hbox{costo(aceptar)} = 0 g(x) + c_a(1-g(x)) = (1-g(x)) c_a
\]
De manera similar, hacemos costo(rechazar):
\[
\ \hbox{costo(rechazar)} = c_r g(x) + 0(1-g(x)) = g(x) c_r
\]
\item Para seleccionar $\kappa$, vemos cuando se igualan los costes de aceptar y rechazar. Substituyendo en las ecuaciones del costo anteriores $g(x)$ por $\kappa$ se obtiene:
\[
\	(1-\kappa) c_a = \kappa c_r \rightarrow c_a = \kappa(c_r+c_a) \rightarrow \kappa = \frac{c_a}{c_a+c_r}
\]
\item En el caso del supermercado, tenemos que $c_a = 1$ y $c_r = 10$. Entonces $\kappa = \frac{1}{11} \simeq 0.09$. Por otro lado, en el caso de la CIA, tenemos que $c_a = 1000$ y $c_r = 1$ y $\kappa = \frac{1000}{1001} \simeq 0.999$.\\

Esto  nos indica que en el caso de la CIA aceptamos a un usuario si la probabilidad de que sea la persona correcta es mayor del 99.9$\%$. Esto es normal, pues la seguridad de la CIA es muy alta y sólo permite pasar si se está muy seguro de que es la persona correcta.\\

En el caso de un supermercado, se puede ser muy permisivo y queremos que el sistema de detección no supongan un impedimento para los usuarios. Por eso, se acepta cuando hay un 10$\%$ o más de probabilidad de que sea la persona correcta. Es decir, sólo se impide el acceso cuando estamos bastante seguros de que no es la persona correcta.\\

\end{enumerate}
\end{respuesta}

\newpage

\end{document}