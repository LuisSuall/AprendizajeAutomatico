%Encabezado estándar
\documentclass[10pt,a4paper]{article}
\usepackage[utf8]{inputenc}

\usepackage{amsmath}
\usepackage{amssymb}
\usepackage{amsthm}
\usepackage{amsfonts}
\usepackage{enumerate}
\usepackage{listings}
\usepackage{hyperref}
\usepackage{eurosym}

\author{Luis Suárez Lloréns}
\title{Cuestionario 2}
\date{}

\theoremstyle{definition}
\newtheorem{cuestion}{Cuestión}
\newtheorem*{bonus}{Bonus}
\newtheorem*{respuesta}{Respuesta}

\begin{document}
\maketitle

%1
\begin{cuestion}
\end{cuestion}
\begin{respuesta}
\textbf{a)}
La clase $\mathcal{H}^1$ tiene un sólo punto que vale 1, y el resto valen 0. Por tanto, es la elección de dicho punto es el parámetro a seleccionar. Por tanto, el número de funciones en $\mathcal{H}^1$ es igual al número de puntos. El número de puntos es $2^{10} = 1024$.\\

La clase $\mathcal{H}^{100}$ tiene por parámetros 100 puntos del espacio. Tenemos que elegir 100 puntos, sin repetición, y sin importar el orden. Por tanto, el número de datos es $\binom{1024}{100}$. Por tanto, tenemos $7.75 \times 10^{140} $ funciones distintas.\\

\textbf{b)}
Simplemente necesitamos almacenar el punto en el que vale 1. Por tanto son 10 bits.\\

\textbf{c)}
De igual manera, necesitamos almacenar los 100 puntos. Cada punto necesita para ser almacenado 10 bits, luego necesitamos 1000 bits.\\

Podemos ver que un espacio de funciones más complejo, no sólo el número de funciones disponibles es mayor, si no que también el de sus componentes. Esto es lógico, pues el número de parámetros crece, y esto hace crecer tanto el número de posibilidades de la clase de funciones como la complejidad de representación de una posible solución, pues necesita dar valor a una variable más.\\
\end{respuesta}

%2
\begin{cuestion}
\end{cuestion}
\begin{respuesta}
\textbf{a)}
Hay 5 partidos, y para cada partido hay dos resultados posibles, ganar o perder. Luego tenemos $2^5$ opciones, es decir 32 predicciones distintas.\\

\textbf{b)}
Evidentemente, si hay 32 opciones, puede enviar a 32 personas distintas una predicción distinta a cada una. Así, al terminar el proceso, una y sólo una de las 32 personas habrá recibido las 5 predicciones correctas.\\

\textbf{c)}
Tendrá que enviar la carta a la mitad de las personas, pues habrá enviado 16 "gana el equipo 1" y 16 "gana el equipo 2".\\

\textbf{d)}
Habrá enviado 32 cartas la primera semana, 16 la segunda, 8 la tercera, 4 la cuarta y 2 la última semana. En total 62 cartas distintas.\\

\textbf{e)}
Simplemente tenemos que realizar la siguiente operación:\\

\[
\ 50000 - 0.5 \times 62 = 49969
\]

Se obtendrían 49969 \euro.\\

\textbf{f)}
TODO\\

\end{respuesta}

%3
\begin{cuestion}
\end{cuestion}
\begin{respuesta}

\end{respuesta}

%4
\begin{cuestion}
\end{cuestion}
\begin{respuesta}
Este procedimiento tiene, de base, un gran problema. Observar los datos contamina de manera total el procedimiento, pues nos lleva directamente a elegir un procedimiento y un conjunto de funciones, en este caso lineales. Esto es incorrecto, pues ya estamos introduciendo conocimiento del problema.\\

Dado esto, hace que no tenga sentido la parte de obtener una cota del error, pues el error en el procedimiento cometido al seleccionar la función o el conjunto de funciones también rompe el concepto de la cota.\\
 
\end{respuesta}

%5
\begin{cuestion}
\end{cuestion}
\begin{respuesta}
\end{respuesta}

%6
\begin{cuestion}
\end{cuestion}
\begin{respuesta}
Este procedimiento tiene varios errores.\\

Por un lado, seleccionar el $k$ para el que el error en la muestra $E_{in}$ es mínima, va a seleccionar siempre $k=1$. Esto es lógico, pues con $k=1$ seleccionamos siempre como clase el más cercano, lo que dentro de la muestra es siempre es el mismo punto que estás evaluando, luego el error $E_{in}$ va a ser 0.\\

Podría pasar que de verdad $k=1$ sea la mejor elección, pero sabemos que no tiene porque serlo. Es más, $k=1$ realiza un gran sobreaprendizaje de la muestra lo que, como sabemos, reduce enormemente la capacidad de generalización.\\
  
Por otro lado, tenemos que no es adecuado de todas formas seleccionar la cota para $N$ funciones. Esto se debe a que cada una de las hipótesis ---cada uno de los distintos $k$--- hay infinitas funciones en esa clase, y no una sola como indica el ejercicio. Es más, para $k=1$, podemos clasificar perfectamente cualquier conjunto de puntos, por lo que hemos visto antes. Esto nos indica que la cota de Vapnik-Chervonenskis es infinita. Es claro que una sola función no puede hacer eso. Luego cada $k$ nos da un conjunto de funciones, y no se puede realizar la cota que se nos indica.\\
  
\end{respuesta}

%7
\begin{cuestion}
\end{cuestion}
\begin{respuesta}
\end{respuesta}

%8
\begin{cuestion}
\end{cuestion}
\begin{respuesta}
\end{respuesta}

%9
\begin{cuestion}
\end{cuestion}
\begin{respuesta}
\end{respuesta}

%10
\begin{cuestion}
\end{cuestion}
\begin{respuesta}
\end{respuesta}

\end{document}